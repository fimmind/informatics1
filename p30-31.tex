\section{Страницы 30 - 31}
\subsection{Задача 1}
По условию, только один поезд едит в Санкт-Петербург, а всего поездов 8, значит 
выбор происходит из восьми различных вариантов. По формуле $ N = 2^{i} $
получаем, что количество информации равно 3 битам, значит пасажир не прав.

\subsection{Задача 2}
В данной ситуации происходит выбор из двую вариантов: либо обезьяна сидит в
первом вальере, либо во втором. По формуле $ N = 2^{i} $, получаем, что 
количество информации равно 1 биту, значит посетитель прав.

\subsection{Задача 3}
Для каждой из четырех пещер происходит выбор из двух вариантов (Либо в нем лежит 
клад, либо нет), так что всего существует $ 2^{4} $ вариантов, значит по 
формуле $ N = 2^{i} $ мы получаем количествоинформации равное 4 битам, значит 
для кодирования сведений о расположении кладов необходимо 4 или более бит.

\subsection{Задача 4}
Для первого клада существует выбор из четырех вариантов расположения, для второго --
из трех, так что всего существует $ 4 \times 3 = 12 $ вариантов расположения
двух кладов. Округляем в большую сторону до степени двойки и по формуле 
$ N = 2^{i} $ получаем $ i = 4 $, тоесть необходимо 4 бита информации.

\subsection{Задача 5}
Аналогично прошлой задаче выбор происходит из 12 вариантов и необходимый 
объем информации -- 4 бита. Каждый отдельный случай можно закодировать 
четырехзначным двоизным числом следующим образом: цифры третьего и второго
разрядов указывают на двоичный номер первого ключа уменьшенный на единицу,
первого и нулевого -- на номер второго в том же формате. Приведенное высказывание 
будет закодировано как число 0111.

\subsection{Задача 14}
$ 8\ \text{Кб} = 2^{3} \times 2^{10} \times 2^{3}\ \text{бит} = 2^{16}\ \text{бит} $

\subsection{Задача 15}
$ \dfrac{1}{16}\ \text{Кб} = 2^{-4} \times 2^{10} \times 2^{3}\ \text{бит} = 2^{9}\ \text{бит} = 512\ \text{бит} $

\subsection{Задача 16}
$ \dfrac{1}{512}\ \text{Мб} = 2^{-9} \times 2^{10} \times 2^{10} \times 2^{3}\ \text{бит} 
  = 2^{14}\ \text{бит} $

