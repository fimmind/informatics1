\section{Страницы 97 - 99}
\subsection{Задача 3}
Из двух чисел с одинаковой записью, но разными основаниями систем счисления,
большим будет число с большим основанием. Исключение -- числа длиной в один 
знак будут равны.

Ответ: $ 11_{25} $

\subsection{Задача 4}
\begin{equation*}
  \begin{gathered}
    345_{6} = 6^2 \times 3 + 6 \times 4 + 5 = 108 + 24 + 5 = 137 \\
    345_{7} = 7^2 \times 3 + 7 \times 4 + 5 = 147 + 28 + 5 = 180 \\
    345_{8} = 8^2 \times 3 + 8 \times 4 + 5 = 192 + 32 + 5 = 229 \\
    345_{9} = 9^2 \times 3 + 9 \times 4 + 5 = 243 + 36 + 5 = 284 \\
  \end{gathered}
\end{equation*}

\subsection{Задача 22}
По схеме Горнера:
\begin{equation*}
  \begin{cases}
    30 = (k_2p + k_1) p + k_0, \\
    p > 0 \\
    0 < k_2 < p, \\
    0 \le k_1 < p, \\
    0 \le k_0 < p, \\
    k_0, k_1, k_2, p \in \mathbb N 
  \end{cases}
\end{equation*}

При целочисленном делении обоих частей равенства на $ p $ получаем:
\begin{equation}
  \label{eq:22div}
  \left\lfloor \frac{30}{p} \right\rfloor = (k_2p + k_1)
\end{equation}

Поскольку $ k_2 > 0 $ и $ k_1 \ge 0 $, то $ p \le k_2p + k_1 $. 
Тогда по равенству \eqref{eq:22div} получаем: 
\begin{equation*}
  \begin{gathered}
    p \le \left\lfloor \frac{30}{p} \right\rfloor \\
    (\forall x: \left\lfloor x \right\rfloor \le x) \Rightarrow p \le \left\lfloor \frac{30}{p} \right\rfloor \le \frac{30}{p}
    \Rightarrow p \le \frac{30}{p} \\
    p > 0 \Rightarrow p^2 \le 30 \Rightarrow p \le \sqrt{30} \\
    p \in \mathbb N \Rightarrow \max p = \left\lfloor \sqrt{30} \right\rfloor = 5
  \end{gathered}
\end{equation*}

Ответ: 5

\subsection{Задача 23}
Аналогично прошлой задаче:
\begin{equation*}
  \max p = \left\lfloor \sqrt{70} \right\rfloor = 8
\end{equation*}

Ответ: 8

\subsection{Задача 30}
Сопоставим каждое слово с некторым числом, в котором в четверичной системе 
счисления каждой из букв А, К, Р и У соответствуют цифры 0, 1, 2 и 3
соответственно.

\begin{enumerate}
  \item $ N = 4^{5} = 2^{10} = 1024 $
  \item 
    \begin{enumerate}
      \item На 150-ом месте стоит слово, которому соответствует число $ 149 = 02111_4 $,
    значит 150-ое слово -- это АРККК
      \item На 250-ом месте стоит слово, которому соответствует число $ 249 = 03321_4 $,
    значит 250-ое слово -- это АУУРК
      \item На 350-ом месте стоит слово, которому соответствует число $ 349 = 11131_4 $,
    значит 350-ое слово -- это КККУК
      \item На 450-ом месте стоит слово, которому соответствует число $ 449 = 13001_4 $,
    значит 450-ое слово -- это КУААК
    \end{enumerate}
  \item
    \begin{enumerate}
      \item Слову АКУРА соответствует число $ 01320_4 = 120 $, значит -- это 121-ое слово.
      \item Слову КАРАУ соответствует число $ 10203_4 = 291 $, значит -- это 292-ое слово.
      \item Слову РУКАА соответствует число $ 23100_4 = 720 $, значит -- это 721-ое слово.
      \item Слову УКАРА соответствует число $ 31020_4 = 840 $, значит -- это 841-ое слово.
      \item Слову УРАКА соответствует число $ 32010_4 = 900 $, значит -- это 901-ое слово.
    \end{enumerate}
  \item
    Первому такому слову соответствует число $ 20000_4 = 512 $, значит -- это 513-ое слово.

    Последнему такому слову соответствует число $ 23333_4 = 767 $, значит -- это 768-ое слово.
\end{enumerate}
