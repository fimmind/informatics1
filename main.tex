\documentclass[12pt, a4paper]{article}

\usepackage[T2A]{fontenc}
\usepackage[utf8]{inputenc}
\usepackage[english, russian]{babel}
\usepackage[left=1.5cm, right=1.5cm, top=3cm, bottom=2cm]{geometry}

% Настройка ссылок
\usepackage[
  unicode, pdftex, 
  colorlinks=true, 
  linkcolor=black, 
  pdfhighlight=/P
  ]{hyperref} 

% Колонтитул
\usepackage{fancyhdr}
\pagestyle{fancy}
\fancyhf{}
\rhead{Виногродский Серафим}
\lhead{Домашние работы за первую четверть}
\rfoot{Стр. \thepage}

\usepackage{amsmath}
\usepackage{amssymb}
\usepackage{hhline}
\usepackage{indentfirst}
% \usepackage{listings}
% \usepackage{ifthen}
% \usepackage{xparse}
\usepackage{tikz}

\title{Домашние работы за первую четверть}
\author{Виногродский Серафим}

\renewcommand{\thesection}{}
\renewcommand{\thesubsection}{}
\renewcommand{\thesubsubsection}{}
\renewcommand{\theparagraph}{}

\tikzset{
  cir/.style = {circle, draw=black}
}

\begin{document}
  \maketitle
  \setcounter{tocdepth}{2}
  \tableofcontents
  \pagebreak

  \part{Задачи}

  \section{Страницы 30 - 31}
  \subsection{Задача 1}
  По условию, только один поезд едит в Санкт-Петербург а всего поездов 8, значит 
  происходит выбор из восьми различных вариантов. По формуле $ N = 2^{i} $
  получаем:
  \begin{equation*}
    \begin{gathered}
      8 = 2^{i} \\
      i = 3
    \end{gathered}
  \end{equation*}

  Количество информации равно 3 битам, значит пасажир не прав.

  \subsection{Задача 2}
  В данной ситуации происходит выбор из двую вариантов: либо обезьяна сидит в
  первом вальере, либо во втором. По формуле $ N = 2^{i} $, получаем:
  \begin{equation*}
    \begin{gathered}
      2 = 2^{i} \\
      i = 1
    \end{gathered}
  \end{equation*}

  Количество информации равно 1 биту, значит посетитель прав.

  \subsection{Задача 3}
  Для каждой из четырех пещер происходит выбор из двух вариантов, так что всего 
  существует $ 2^{4} $ вариантов, значит по формуле $ N = 2^{i} $ мы получаем 
  количествоинформации равное 4 битам, значит для кодирования сведений о расположении
  кладов необходимо 4 или более бит.

  \subsection{Задача 4}
  Для первого клада существует выбор из четырех вариантов расположения, для второго --
  из трех, так что всего существует $ 4 \times 3 = 12 $ вариантов расположения
  двух кладов. Округляем в большую сторону до степени двойки и по формуле 
  $ N = 2^{i} $ получаем $ i = 4 $, тоесть необходимо 4 бита информации.
  
  \subsection{Задача 5}
  Аналогично прошлой задаче выбор происходит из 12 вариантов и необходимый 
  обьем информации -- 4 бита. Каждый отдельный случай можно закодировать 
  четырехзначным двоизным числом следующим образом: первые две цифры указывают 
  на двоичный номер первого ключа уменьшенный на единицу, последние две -- на 
  номер второго в том же формате. Приведенное высказывание будет закодировано 
  как число 0111.

  \subsection{Задача 14}
  $ 8\ \text{Кб} = 2^{3} \times 2^{10} \times 2^{3}\ \text{бит} = 2^{16}\ \text{бит} $

  \subsection{Задача 15}
  $ \dfrac{1}{16}\ \text{Кб} = 2^{-4} \times 2^{10} \times 2^{3}\ \text{бит} = 2^{9}\ \text{бит} $

  \subsection{Задача 16}
  $ \dfrac{1}{512}\ \text{Мб} = 2^{-9} \times 2^{10} \times 2^{10} \times 2^{3}\ \text{бит} 
    = 2^{14}\ \text{бит} $
  

  \section{Страницы 48 - 49}
  \subsection{№1}
  Структурирование информации -- это процесс ее организации.
  Оно используется для облегчения ее восприятия и доступа к ней.

  \subsection{№2}
  Алфавитный (Лексикографический) порядок -- порядок сортировки упорядоченных 
  контейнеров при котором для сравнения используется следующий алгоритм:
  Если длины контейнеров не одинаковы, то меньшим считается контейнер меньшей 
  длины. В ином случае сравниваются первые элементы контейнеров, и, если они 
  равны, то алгоритм повторяется для хвостов контейнеров. В ином случае 
  использьзуется результат сравнения элементов. Пустые контейнеры полагаются 
  равными.

  Типичным примером является порядок слов в словаре.

  \subsection{№8}
  Записать соответствие между номерами столбцов и строк и данными, находящимися
  в ячейке.

  \subsection{№9}
  Дерево (Иерархия) -- это связаный ациклический граф.

  Направленное дерево -- это связанный ациклический орграф, в котором любая вершина 
  является конечной только для одной дуги.

  \section{Страницы 74 - 75}
  \subsection{Задача 5}
  Ответ: ББААВА, БГАВА

  \subsection{Задача 6}
  Ответ: АДААВВ, АВГАВВ, АВВВААВВ, АВВБАВВ

  \subsection{Задача 9}
  Построение дерева по кодовой таблице показывает, что наименьшее значения для Г,
  при котором выполнеяется условие Фано -- это 111.

  \subsection{Задача 10}
  Построение дерева по кодовой таблице показывает, что наименьшее значения для Г,
  при котором выполнеяется условие Фано -- это 11.

  \section{Страницы 97 - 99}
  \subsection{Задача 3}
  Из двух чисел с одинаковой записью, но разными основаниями систем счисления,
  большим будет то, у которого больше основание. Исключение -- числа длиной в 
  один знак будут равны.

  Ответ: $ 11_{25} $

  \subsection{Задача 4}
  \begin{equation*}
    \begin{gathered}
      345_{6} = 6^2 \times 3 + 6 \times 4 + 5 = 108 + 24 + 5 = 137 \\
      345_{7} = 7^2 \times 3 + 7 \times 4 + 5 = 147 + 28 + 5 = 180 \\
      345_{8} = 8^2 \times 3 + 8 \times 4 + 5 = 192 + 32 + 5 = 229 \\
      345_{9} = 9^2 \times 3 + 9 \times 4 + 5 = 243 + 36 + 5 = 284 \\
    \end{gathered}
  \end{equation*}

  \subsection{Задача 22}
  По схеме Горнера:
  \begin{equation}
    \label{eq:22gorner}
    \begin{cases}
      30 = (k_2p + k_1) p + k_0, \\
      0 < k_2 < p, \\
      0 \le k_1 < p, \\
      0 \le k_0 < p, \\
      k_0, k_1, k_2 \in \mathbb N 
    \end{cases}
  \end{equation}
 
  При целочисленном делении обоих частей равенства на $ p $ получаем:
  \begin{equation}
    \label{eq:22div}
    \left\lfloor \frac{30}{p} \right\rfloor = (k_2p + k_1)
  \end{equation}

  Поскольку $ k_2 > 0 $ и $ k_1 \ge 0 $, то $ p \le k_2p + k_1 $. 
  Тогда по равенству \eqref{eq:22div} получаем: 
  \begin{equation*}
    p \le \left\lfloor \frac{30}{p} \right\rfloor \Longleftrightarrow p \le \left\lfloor \sqrt{30} \right\rfloor
  \end{equation*}

  Наибольшее значение $ p = \left\lfloor \sqrt{30} \right\rfloor = \sqrt{25} = 5 $

  Ответ: 5

  \subsection{Задача 23}
  Аналогично прошлой задаче:
  \begin{equation*}
    p = \left\lfloor \sqrt{70} \right\rfloor = \sqrt{64} = 8
  \end{equation*}

  Ответ: 8

  \subsection{Задача 30}
  Сопоставим каждое слово с некторым числом, в котором в четверичной системе 
  счисления каждой из букв А, К, Р и У соответствуют цифры 0, 1, 2 и 3
  соответственно.

  \begin{enumerate}
    \item $ N = 4^{5} = 2^{10} = 1024 $
    \item 
      \begin{enumerate}
        \item На 150-ом месте стоит слово, которому соответствует число $ 149 = 02111_4 $,
      значит 150-ое слово -- это АРККК
        \item На 250-ом месте стоит слово, которому соответствует число $ 249 = 03321_4 $,
      значит 250-ое слово -- это АУУРК
        \item На 350-ом месте стоит слово, которому соответствует число $ 349 = 11131_4 $,
      значит 350-ое слово -- это КККУК
        \item На 450-ом месте стоит слово, которому соответствует число $ 449 = 13001_4 $,
      значит 450-ое слово -- это КУААК
      \end{enumerate}
    \item
      \begin{enumerate}
        \item Слову АКУРА соответствует число $ 01320_4 = 120 $, значит -- это 121-ое слово.
        \item Слову КАРАУ соответствует число $ 10203_4 = 291 $, значит -- это 292-ое слово.
        \item Слову РУКАА соответствует число $ 23100_4 = 720 $, значит -- это 721-ое слово.
        \item Слову УКАРА соответствует число $ 31020_4 = 840 $, значит -- это 841-ое слово.
        \item Слову УРАКА соответствует число $ 32010_4 = 900 $, значит -- это 901-ое слово.
      \end{enumerate}
    \item
      Первому такому слову соответствует число $ 20000_4 = 512 $, значит -- это 513-ое слово.

      Последнему такому слову соответствует число $ 23333_4 = 767 $, значит -- это 768-ое слово.
  \end{enumerate}

  \section{Страницы 176 - 177}
  \subsection{Задача 1}

  \subsubsection{а)}

  \begin{tikzpicture}[level/.style={sibling distance=60mm/#1}, scale=0.6]
  \node [cir] {$ \lor $}
    child { 
      node [cir] {$ \land $} 
      child {node [cir] {A}} 
      child {node [cir] {B}}
      edge from parent node[above left] {$ \lnot $}
    }
    child { 
      node [cir] {$ \land $}
      child {node [cir] {A}}
      child {node [cir] {B}}
    };
  \end{tikzpicture}

  \rule{0cm}{0cm}

  \begin{tabular}{|c|c|c|}
    \hline
    A & B & $ \lnot (A \land B) \lor (A \land B) $ \\\hline
    0 & 0 & 1 \\\hline
    0 & 1 & 1 \\\hline
    1 & 0 & 1 \\\hline
    1 & 1 & 1 \\\hline
  \end{tabular}

  \subsubsection{б)}

  \begin{tikzpicture}[level/.style={sibling distance=87mm/#1}, scale=0.6]
  \node [circle, draw=black] {$ \lor $}
    child { 
      node [circle, draw=black] {$ \land $}
      child { node [circle, draw=black] {A} }
      child { node [circle, draw=black] {B} }
    }
    child {
      node [circle, draw=black] {$ \lor $}
      child { 
        node [circle, draw=black] {$ \land $} 
        child { 
          node [circle, draw=black] {A} 
          edge from parent node[above left] {$ \lnot $}
        }
        child { 
          node [circle, draw=black] {B} 
          edge from parent node[above right] {$ \lnot $}
        }
      }
      child { 
        node [circle, draw=black] {$ \land $} 
        child { node [circle, draw=black] {A} }
        child { 
          node [circle, draw=black] {B} 
          edge from parent node[above right] {$ \lnot $}
        }
      }
    };
  \end{tikzpicture}

  \rule{0cm}{0cm}

  \begin{tabular}{|c|c|c|}
    \hline
    A & B & $ A \land B \lor \lnot A \land \lnot B \lor A \land \lnot B $ \\\hline
    0 & 0 & 1 \\\hline
    0 & 1 & 0 \\\hline
    1 & 0 & 1 \\\hline
    1 & 1 & 1 \\\hline
  \end{tabular}

  \subsubsection{в)}

  \begin{tikzpicture}[level/.style={sibling distance=87mm/#1}, scale=0.6]
  \node [circle, draw=black] {$ \land  $}
    child { 
      node [circle, draw=black] {$ \lor  $}
      child {
        node [circle, draw=black] {A}
      }
      child {
        node [circle, draw=black] {B}
      }
    }
    child {
      node [circle, draw=black] {$ \land  $} child {
        node [circle, draw=black] {$ \lor  $}
        child {
          node [circle, draw=black] {A}
          edge from parent node[above left] {$ \lnot  $}
        }
        child {
          node [circle, draw=black] {B}
          edge from parent node[above right] {$ \lnot  $}
        }
      }
      child {
        node [circle, draw=black] {$ \lor  $}
        child {
          node [circle, draw=black] {A}
        }
        child {
          node [circle, draw=black] {B}
          edge from parent node[above right] {$ \lnot  $}
        }
      }
    };
  \end{tikzpicture}

  \rule{0cm}{0cm}

  \begin{tabular}{|c|c|c|}
    \hline
    A & B & $ (A \lor B) \land (\lnot A \lor \lnot B) \land (A \lor \lnot B) $ \\\hline
    0 & 0 & 0 \\\hline
    0 & 1 & 0 \\\hline
    1 & 0 & 1 \\\hline
    1 & 1 & 0 \\\hline
  \end{tabular}

  \subsubsection{г)}
  \begin{tikzpicture}[level/.style={sibling distance=87mm/#1}, scale=0.6]
  \node [circle, draw=black] {$ \lor  $}
    child { 
      node [circle, draw=black] {$ \land  $}
      child {
        node [circle, draw=black] {A}
      }
      child {
        node [circle, draw=black] {B}
        edge from parent node[above right] {$ \lnot  $}
      }
    }
    child {
      node [circle, draw=black] {$ \lor  $}
      child {
        node [circle, draw=black] {$ \land $}
        child {
          node [circle, draw=black] {B}
        }
        child {
          node [circle, draw=black] {C}
          edge from parent node[above right] {$ \lnot  $}
        }
      }
      child {
        node [circle, draw=black] {$ \land  $}
        child {
          node [circle, draw=black] {C}
        }
        child {
          node [circle, draw=black] {A}
          edge from parent node[above right] {$ \lnot  $}
        }
      }
    };
  \end{tikzpicture}

  \begin{tabular}{|c|c|c|c|}
    \hline
    A & B & C & $ A \land \lnot B \lor B \land \lnot C \lor C \land \lnot A $ \\\hline
    0 & 0 & 0 & 0 \\\hline
    0 & 0 & 1 & 1 \\\hline
    0 & 1 & 0 & 1 \\\hline
    0 & 1 & 1 & 1 \\\hline
    1 & 0 & 0 & 1 \\\hline
    1 & 0 & 1 & 1 \\\hline
    1 & 1 & 0 & 1 \\\hline
    1 & 1 & 1 & 0 \\\hline
  \end{tabular}

  \subsubsection{д)}

  \pagebreak

  \part{Сообщение}
  \section{<<Графы в практических задачах>>}
  
\end{document}
