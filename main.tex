\documentclass[12pt, a4paper]{article}

\usepackage[T2A]{fontenc}
\usepackage[utf8]{inputenc}
\usepackage[english, russian]{babel}
\usepackage[left=1.5cm, right=1.5cm, top=3cm, bottom=2cm]{geometry}

% Настройка ссылок
\usepackage[
  unicode, pdftex, 
  colorlinks=true, 
  linkcolor=black, 
  pdfhighlight=/P
  ]{hyperref} 

% Колонтитул
\usepackage{fancyhdr}
\pagestyle{fancy}
\fancyhf{}
\rhead{Виногродский Серафим}
\lhead{Домашние работы за первую четверть}
\rfoot{Стр. \thepage}

\usepackage{amsmath}
\usepackage{amssymb}
\usepackage{hhline}
\usepackage{indentfirst}
% \usepackage{listings}
% \usepackage{ifthen}
% \usepackage{xparse}
\usepackage{tikz}

\title{Домашние работы за первую четверть}
\author{Виногродский Серафим}

\renewcommand{\thesection}{}
\renewcommand{\thesubsection}{}
\renewcommand{\thesubsubsection}{}
\renewcommand{\theparagraph}{}

\tikzset{
  cir/.style = {circle, draw=black}
}

\begin{document}
  \maketitle
  \setcounter{tocdepth}{2}
  \tableofcontents
  \pagebreak

  \part{Задачи}

  \section{Страницы 30 - 31}
  \subsection{Задача 1}
  По условию, только один поезд едит в Санкт-Петербург а всего поездов 8, значит 
  происходит выбор из восьми различных вариантов. По формуле $ N = 2^{i} $
  получаем:
  \begin{equation*}
    \begin{gathered}
      8 = 2^{i} \\
      i = 3
    \end{gathered}
  \end{equation*}

  Количество информации равно 3 битам, значит пасажир не прав.

  \subsection{Задача 2}
  В данной ситуации происходит выбор из двую вариантов: либо обезьяна сидит в
  первом вальере, либо во втором. По формуле $ N = 2^{i} $, получаем:
  \begin{equation*}
    \begin{gathered}
      2 = 2^{i} \\
      i = 1
    \end{gathered}
  \end{equation*}

  Количество информации равно 1 биту, значит посетитель прав.

  \subsection{Задача 3}
  Для каждой из четырех пещер происходит выбор из двух вариантов, так что всего 
  существует $ 2^{4} $ вариантов, значит по формуле $ N = 2^{i} $ мы получаем 
  количествоинформации равное 4 битам, значит для кодирования сведений о расположении
  кладов необходимо 4 или более бит.

  \subsection{Задача 4}
  Для первого клада существует выбор из четырех вариантов расположения, для второго --
  из трех, так что всего существует $ 4 \times 3 = 12 $ вариантов расположения
  двух кладов. Округляем в большую сторону до степени двойки и по формуле 
  $ N = 2^{i} $ получаем $ i = 4 $, тоесть необходимо 4 бита информации.
  
  \subsection{Задача 5}
  Аналогично прошлой задаче выбор происходит из 12 вариантов и необходимый 
  обьем информации -- 4 бита. Каждый отдельный случай можно закодировать 
  четырехзначным двоизным числом следующим образом: первые две цифры указывают 
  на двоичный номер первого ключа уменьшенный на единицу, последние две -- на 
  номер второго в том же формате. Приведенное высказывание будет закодировано 
  как число 0111.

  \subsection{Задача 14}
  $ 8\ \text{Кб} = 2^{3} \times 2^{10} \times 2^{3}\ \text{бит} = 2^{16}\ \text{бит} $

  \subsection{Задача 15}
  $ \dfrac{1}{16}\ \text{Кб} = 2^{-4} \times 2^{10} \times 2^{3}\ \text{бит} = 2^{9}\ \text{бит} $

  \subsection{Задача 16}
  $ \dfrac{1}{512}\ \text{Мб} = 2^{-9} \times 2^{10} \times 2^{10} \times 2^{3}\ \text{бит} 
    = 2^{14}\ \text{бит} $
  

  \section{Страницы 48 - 49}
  \subsection{№1}
  Структурирование информации -- это процесс ее организации.
  Оно используется для облегчения ее восприятия и доступа к ней.

  \subsection{№2}
  Алфавитный (Лексикографический) порядок -- порядок сортировки упорядоченных 
  контейнеров при котором для сравнения используется следующий алгоритм:
  Если длины контейнеров не одинаковы, то меньшим считается контейнер меньшей 
  длины. В ином случае сравниваются первые элементы контейнеров, и, если они 
  равны, то алгоритм повторяется для хвостов контейнеров. В ином случае 
  использьзуется результат сравнения элементов. Пустые контейнеры полагаются 
  равными.

  Типичным примером является порядок слов в словаре.

  \subsection{№8}
  Записать соответствие между номерами столбцов и строк и данными, находящимися
  в ячейке.

  \subsection{№9}
  Дерево (Иерархия) -- это связаный ациклический граф.

  Направленное дерево -- это связанный ациклический орграф, в котором любая вершина 
  является конечной только для одной дуги.

  \section{Страницы 74 - 75}
  \subsection{Задача 5}
  Ответ: ББААВА, БГАВА

  \subsection{Задача 6}
  Ответ: АДААВВ, АВГАВВ, АВВВААВВ, АВВБАВВ

  \subsection{Задача 9}
  Построение дерева по кодовой таблице показывает, что наименьшее значения для Г,
  при котором выполнеяется условие Фано -- это 111.

  Ответ: 111

  \subsection{Задача 10}
  Построение дерева по кодовой таблице показывает, что наименьшее значения для Г,
  при котором выполнеяется условие Фано -- это 11.

  Ответ: 11

  \section{Страницы 97 - 99}
  \subsection{Задача 3}
  Из двух чисел с одинаковой записью, но разными основаниями систем счисления,
  большим будет то, у которого больше основание. Исключение -- числа длиной в 
  один знак будут равны.

  Ответ: $ 11_{25} $

  \subsection{Задача 4}
  \begin{equation*}
    \begin{gathered}
      345_{6} = 6^2 \times 3 + 6 \times 4 + 5 = 108 + 24 + 5 = 137 \\
      345_{7} = 7^2 \times 3 + 7 \times 4 + 5 = 147 + 28 + 5 = 180 \\
      345_{8} = 8^2 \times 3 + 8 \times 4 + 5 = 192 + 32 + 5 = 229 \\
      345_{9} = 9^2 \times 3 + 9 \times 4 + 5 = 243 + 36 + 5 = 284 \\
    \end{gathered}
  \end{equation*}

  \subsection{Задача 22}
  По схеме Горнера:
  \begin{equation}
    \label{eq:22gorner}
    \begin{cases}
      30 = (k_2p + k_1) p + k_0, \\
      0 < k_2 < p, \\
      0 \le k_1 < p, \\
      0 \le k_0 < p, \\
      k_0, k_1, k_2 \in \mathbb N 
    \end{cases}
  \end{equation}
 
  При целочисленном делении обоих частей равенства на $ p $ получаем:
  \begin{equation}
    \label{eq:22div}
    \left\lfloor \frac{30}{p} \right\rfloor = (k_2p + k_1)
  \end{equation}

  Поскольку $ k_2 > 0 $ и $ k_1 \ge 0 $, то $ p \le k_2p + k_1 $. 
  Тогда по равенству \eqref{eq:22div} получаем: 
  \begin{equation*}
    p \le \left\lfloor \frac{30}{p} \right\rfloor \Longleftrightarrow p \le \left\lfloor \sqrt{30} \right\rfloor
  \end{equation*}

  Наибольшее значение $ p = \left\lfloor \sqrt{30} \right\rfloor = \sqrt{25} = 5 $

  Ответ: 5

  \subsection{Задача 23}
  Аналогично прошлой задаче:
  \begin{equation*}
    p = \left\lfloor \sqrt{70} \right\rfloor = \sqrt{64} = 8
  \end{equation*}

  Ответ: 8

  \subsection{Задача 30}
  Сопоставим каждое слово с некторым числом, в котором в четверичной системе 
  счисления каждой из букв А, К, Р и У соответствуют цифры 0, 1, 2 и 3
  соответственно.

  \begin{enumerate}
    \item $ N = 4^{5} = 2^{10} = 1024 $
    \item 
      \begin{enumerate}
        \item На 150-ом месте стоит слово, которому соответствует число $ 149 = 02111_4 $,
      значит 150-ое слово -- это АРККК
        \item На 250-ом месте стоит слово, которому соответствует число $ 249 = 03321_4 $,
      значит 250-ое слово -- это АУУРК
        \item На 350-ом месте стоит слово, которому соответствует число $ 349 = 11131_4 $,
      значит 350-ое слово -- это КККУК
        \item На 450-ом месте стоит слово, которому соответствует число $ 449 = 13001_4 $,
      значит 450-ое слово -- это КУААК
      \end{enumerate}
    \item
      \begin{enumerate}
        \item Слову АКУРА соответствует число $ 01320_4 = 120 $, значит -- это 121-ое слово.
        \item Слову КАРАУ соответствует число $ 10203_4 = 291 $, значит -- это 292-ое слово.
        \item Слову РУКАА соответствует число $ 23100_4 = 720 $, значит -- это 721-ое слово.
        \item Слову УКАРА соответствует число $ 31020_4 = 840 $, значит -- это 841-ое слово.
        \item Слову УРАКА соответствует число $ 32010_4 = 900 $, значит -- это 901-ое слово.
      \end{enumerate}
    \item
      Первому такому слову соответствует число $ 20000_4 = 512 $, значит -- это 513-ое слово.

      Последнему такому слову соответствует число $ 23333_4 = 767 $, значит -- это 768-ое слово.
  \end{enumerate}

  \section{Страницы 176 - 177}
  \subsection{Задача 1}

  \subsubsection{а)}
  \begin{tikzpicture}[level/.style={sibling distance=60mm/#1}, scale=0.6]
  \node [cir] {$ \lor $}
    child { 
      node [cir] {$ \land $} 
      child {node [cir] {A}} 
      child {node [cir] {B}}
      edge from parent node[above left] {$ \lnot $}
    }
    child { 
      node [cir] {$ \land $}
      child {node [cir] {A}}
      child {node [cir] {B}}
    };
  \end{tikzpicture}

  \rule{0cm}{0cm}

  \begin{tabular}{|c|c|c|}
    \hline
    A & B & $ \lnot (A \land B) \lor (A \land B) $ \\\hline
    0 & 0 & 1 \\\hline
    0 & 1 & 1 \\\hline
    1 & 0 & 1 \\\hline
    1 & 1 & 1 \\\hline
  \end{tabular}


  \subsubsection{б)}
  \begin{tikzpicture}[level/.style={sibling distance=87mm/#1}, scale=0.6]
  \node [circle, draw=black] {$ \lor $}
    child { 
      node [circle, draw=black] {$ \land $}
      child { node [circle, draw=black] {A} }
      child { node [circle, draw=black] {B} }
    }
    child {
      node [circle, draw=black] {$ \lor $}
      child { 
        node [circle, draw=black] {$ \land $} 
        child { 
          node [circle, draw=black] {A} 
          edge from parent node[above left] {$ \lnot $}
        }
        child { 
          node [circle, draw=black] {B} 
          edge from parent node[above right] {$ \lnot $}
        }
      }
      child { 
        node [circle, draw=black] {$ \land $} 
        child { node [circle, draw=black] {A} }
        child { 
          node [circle, draw=black] {B} 
          edge from parent node[above right] {$ \lnot $}
        }
      }
    };
  \end{tikzpicture}

  \rule{0cm}{0cm}

  \begin{tabular}{|c|c|c|}
    \hline
    A & B & $ A \land B \lor \lnot A \land \lnot B \lor A \land \lnot B $ \\\hline
    0 & 0 & 1 \\\hline
    0 & 1 & 0 \\\hline
    1 & 0 & 1 \\\hline
    1 & 1 & 1 \\\hline
  \end{tabular}


  \subsubsection{в)}
  \begin{tikzpicture}[level/.style={sibling distance=87mm/#1}, scale=0.6]
  \node [circle, draw=black] {$ \land  $}
    child { 
      node [circle, draw=black] {$ \lor  $}
      child {
        node [circle, draw=black] {A}
      }
      child {
        node [circle, draw=black] {B}
      }
    }
    child {
      node [circle, draw=black] {$ \land  $} child {
        node [circle, draw=black] {$ \lor  $}
        child {
          node [circle, draw=black] {A}
          edge from parent node[above left] {$ \lnot  $}
        }
        child {
          node [circle, draw=black] {B}
          edge from parent node[above right] {$ \lnot  $}
        }
      }
      child {
        node [circle, draw=black] {$ \lor  $}
        child {
          node [circle, draw=black] {A}
        }
        child {
          node [circle, draw=black] {B}
          edge from parent node[above right] {$ \lnot  $}
        }
      }
    };
  \end{tikzpicture}

  \rule{0cm}{0cm}

  \begin{tabular}{|c|c|c|}
    \hline
    A & B & $ (A \lor B) \land (\lnot A \lor \lnot B) \land (A \lor \lnot B) $ \\\hline
    0 & 0 & 0 \\\hline
    0 & 1 & 0 \\\hline
    1 & 0 & 1 \\\hline
    1 & 1 & 0 \\\hline
  \end{tabular}


  \subsubsection{г)}
  \begin{tikzpicture}[level/.style={sibling distance=87mm/#1}, scale=0.6]
  \node [circle, draw=black] {$ \lor  $}
    child { 
      node [circle, draw=black] {$ \land  $}
      child {
        node [circle, draw=black] {A}
      }
      child {
        node [circle, draw=black] {B}
        edge from parent node[above right] {$ \lnot  $}
      }
    }
    child {
      node [circle, draw=black] {$ \lor  $}
      child {
        node [circle, draw=black] {$ \land $}
        child {
          node [circle, draw=black] {B}
        }
        child {
          node [circle, draw=black] {C}
          edge from parent node[above right] {$ \lnot  $}
        }
      }
      child {
        node [circle, draw=black] {$ \land  $}
        child {
          node [circle, draw=black] {C}
        }
        child {
          node [circle, draw=black] {A}
          edge from parent node[above right] {$ \lnot  $}
        }
      }
    };
  \end{tikzpicture}

  \begin{tabular}{|c|c|c|c|}
    \hline
    A & B & C & $ A \land \lnot B \lor B \land \lnot C \lor C \land \lnot A $ \\\hline
    0 & 0 & 0 & 0 \\\hline
    0 & 0 & 1 & 1 \\\hline
    0 & 1 & 0 & 1 \\\hline
    0 & 1 & 1 & 1 \\\hline
    1 & 0 & 0 & 1 \\\hline
    1 & 0 & 1 & 1 \\\hline
    1 & 1 & 0 & 1 \\\hline
    1 & 1 & 1 & 0 \\\hline
  \end{tabular}


  \subsubsection{д)}
  \begin{tikzpicture}[level/.style={sibling distance=100mm/#1}, scale=0.6]
  \node [circle, draw=black] {$ \lor  $}
    child { 
      node [circle, draw=black] {$ \land  $}
      child {
        node [circle, draw=black] {$ \land  $}
        child {
          node [circle, draw=black] {A}
        }
        child {
          node [circle, draw=black] {B}
          edge from parent node[above right] {$ \lnot  $}
        }
      }
      child {
        node [circle, draw=black] {C}
      }
    }
    child {
      node [circle, draw=black] {$ \lor  $}
      child {
        node [circle, draw=black] {$ \land  $}
        child {
          node [circle, draw=black] {$ \land  $}
          child {
            node [circle, draw=black] {A}
            edge from parent node[above left] {$ \lnot  $}
          }
          child {
            node [circle, draw=black] {B}
          }
        }
        child {
          node [circle, draw=black] {C}
          edge from parent node[above right] {$ \lnot  $}
        }
      }
      child {
        node [circle, draw=black] {$ \land  $}
        child {
          node [circle, draw=black] {B}
        }
        child {
          node [circle, draw=black] {C}
        }
      }
    };
  \end{tikzpicture}

  \rule{0cm}{0cm}

  \begin{tabular}{|c|c|c|c|}
    \hline
    A & B & C & $ A \land \lnot B \land C \lor \lnot A \land B \land \lnot C \lor B \land C $ \\\hline
    0 & 0 & 0 & 0 \\\hline
    0 & 0 & 1 & 0 \\\hline
    0 & 1 & 0 & 1 \\\hline
    0 & 1 & 1 & 1 \\\hline
    1 & 0 & 0 & 0 \\\hline
    1 & 0 & 1 & 1 \\\hline
    1 & 1 & 0 & 0 \\\hline
    1 & 1 & 1 & 1 \\\hline
  \end{tabular}


  \subsubsection{е)}
  \begin{tikzpicture}[level/.style={sibling distance=100mm/#1}, scale=0.6]
  \node [circle, draw=black] {$ \land  $}
    child { 
      node [circle, draw=black] {A}
    }
    child {
      node [circle, draw=black] {$ \land  $}
      child {
        node [circle, draw=black] {$ \lor  $}
        child {
          node [circle, draw=black] {$ \land  $}
          child {
            node [circle, draw=black] {B}
            edge from parent node[above left] {$ \lnot  $}
          }
          child {
            node [circle, draw=black] {C}
          }
        }
        child {
          node [circle, draw=black] {A}
          edge from parent node[above right] {$ \lnot  $}
        }
      }
      child {
        node [circle, draw=black] {$ \lor  $}
        child {
          node [circle, draw=black] {C}
          edge from parent node[above left] {$ \lnot  $}
        }
        child {
          node [circle, draw=black] {B}
        }
      }
    };
  \end{tikzpicture}

  \rule{0cm}{0cm}

  \begin{tabular}{|c|c|c|c|}
    \hline
    A & B & C & $ A \land (\lnot B \land C \lor \lnot A) \land (\lnot C \lor B) $ \\\hline
    0 & 0 & 0 & 0 \\\hline
    0 & 0 & 1 & 0 \\\hline
    0 & 1 & 0 & 0 \\\hline
    0 & 1 & 1 & 0 \\\hline
    1 & 0 & 0 & 0 \\\hline
    1 & 0 & 1 & 0 \\\hline
    1 & 1 & 0 & 0 \\\hline
    1 & 1 & 1 & 0 \\\hline
  \end{tabular}


  \subsubsection{ж)}
  \begin{tikzpicture}[level/.style={sibling distance=100mm/#1}, scale=0.6]
  \node [circle, draw=black] {$ \lor  $}
    child { 
      node [circle, draw=black] {$ \land  $}
      child {
        node [circle, draw=black] {A}
      }
      child {
        node [circle, draw=black] {C}
      }
      edge from parent node[above left] {$ \lnot  $}
    }
    child {
      node [circle, draw=black] {$ \land  $}
      child {
        node [circle, draw=black] {B}
      }
      child {
        node [circle, draw=black] {C}
        edge from parent node[above right] {$ \lnot  $}
      }
      edge from parent node[above right] {$ \lnot  $}
    };
  \end{tikzpicture}

  \rule{0cm}{0cm}

  \begin{tabular}{|c|c|c|c|}
    \hline
    A & B & C & $ \lnot (A \land B) \lor \lnot (B \land \lnot C) $ \\\hline
    0 & 0 & 0 & 1 \\\hline
    0 & 0 & 1 & 1 \\\hline
    0 & 1 & 0 & 1 \\\hline
    0 & 1 & 1 & 1 \\\hline
    1 & 0 & 0 & 1 \\\hline
    1 & 0 & 1 & 1 \\\hline
    1 & 1 & 0 & 0 \\\hline
    1 & 1 & 1 & 1 \\\hline
  \end{tabular}

  \subsubsection{з)}
  \begin{tikzpicture}[level/.style={sibling distance=100mm/#1}, scale=0.6]
  \node [circle, draw=black] {$ \lor  $}
    child { 
      node [circle, draw=black] {$ \lor  $}
      child {
        node [circle, draw=black] {A}
      }
      child {
        node [circle, draw=black] {C}
      }
      edge from parent node[above left] {$ \lnot  $}
    }
    child {
      node [circle, draw=black] {$ \lor  $}
      child {
        node [circle, draw=black] {B}
      }
      child {
        node [circle, draw=black] {C}
        edge from parent node[above right] {$ \lnot  $}
      }
      edge from parent node[above right] {$ \lnot  $}
    };
  \end{tikzpicture}

  \rule{0cm}{0cm}

  \begin{tabular}{|c|c|c|c|}
    \hline
    A & B & C & $ \lnot (A \lor B) \lor \lnot (B \lor \lnot C) $ \\\hline
    0 & 0 & 0 & 1 \\\hline
    0 & 0 & 1 & 1 \\\hline
    0 & 1 & 0 & 0 \\\hline
    0 & 1 & 1 & 0 \\\hline
    1 & 0 & 0 & 0 \\\hline
    1 & 0 & 1 & 1 \\\hline
    1 & 1 & 0 & 0 \\\hline
    1 & 1 & 1 & 0 \\\hline
  \end{tabular}


  \subsubsection{и)}

  \begin{tikzpicture}[level/.style={sibling distance=180mm/#1}, scale=0.6]
  \node [circle, draw=black] {$ \lnot  $}
    child { 
      node [circle, draw=black] {$ \land  $}
      child { 
        node [circle, draw=black] {$ \land  $}
        child {
          node [circle, draw=black] {A}
          edge from parent node[above left] {$ \lnot  $}
        }
        child {
          node [circle, draw=black] {C}
        }
        edge from parent node[above left] {$ \lnot  $}
      }
      child { 
        node [circle, draw=black] {$ \land  $}
        child {
          node [circle, draw=black] {B}
          edge from parent node[above left] {$ \lnot  $}
        }
        child {
          node [circle, draw=black] {C}
        }
        edge from parent node[above right] {$ \lnot  $}
      }
    };
  \end{tikzpicture}

  \rule{0cm}{0cm}

  \begin{tabular}{|c|c|c|c|}
    \hline
    A & B & C & $ \lnot (\lnot (\lnot A \land C) \land \lnot (\lnot B \land C)) $ \\\hline
    0 & 0 & 0 & 0 \\\hline
    0 & 0 & 1 & 1 \\\hline
    0 & 1 & 0 & 0 \\\hline
    0 & 1 & 1 & 1 \\\hline
    1 & 0 & 0 & 0 \\\hline
    1 & 0 & 1 & 1 \\\hline
    1 & 1 & 0 & 0 \\\hline
    1 & 1 & 1 & 0 \\\hline
  \end{tabular}

  \subsubsection{к)}
  \begin{tikzpicture}[level/.style={sibling distance=100mm/#1}, scale=0.6]
  \node [circle, draw=black] {$ \lor  $}
    child { 
      node [circle, draw=black] {$ \land  $}
      child {
        node [circle, draw=black] {A}
      }
      child {
        node [circle, draw=black] {$ \lor $}
        child {
          node [circle, draw=black] {C}
        }
        child {
          node [circle, draw=black] {$ \land  $}
          child {
            node [circle, draw=black] {B}
          }
          child {
            node [circle, draw=black] {C}
            edge from parent node[above right] {$ \lnot  $}
          }
        }
      }
    }
    child {
      node [circle, draw=black] {$ \land  $}
      child {
        node [circle, draw=black] {C}
      }
      child {
        node [circle, draw=black] {$ \lor  $}
        child {
          node [circle, draw=black] {A}
        }
        child {
          node [circle, draw=black] {B}
        }
        edge from parent node[above right] {$ \lnot  $}
      }
    };
  \end{tikzpicture}

  \rule{0cm}{0cm}

  \begin{tabular}{|c|c|c|c|}
    \hline
    A & B & C & $ A \land (C \lor B \land \lnot C) \lor C \land \lnot (A \lor B) $ \\\hline
    0 & 0 & 0 & 0 \\\hline
    0 & 0 & 1 & 1 \\\hline
    0 & 1 & 0 & 0 \\\hline
    0 & 1 & 1 & 0 \\\hline
    1 & 0 & 0 & 0 \\\hline
    1 & 0 & 1 & 1 \\\hline
    1 & 1 & 0 & 1 \\\hline
    1 & 1 & 1 & 1 \\\hline
  \end{tabular}


  \subsubsection{л)}
  \begin{tikzpicture}[level/.style={sibling distance=100mm/#1}, scale=0.6]
  \node [circle, draw=black] {$ \lor  $}
    child { 
      node [circle, draw=black] {$ \land  $}
      child {
        node [circle, draw=black] {A}
      }
      child {
        node [circle, draw=black] {$ \lor  $}
        child {
          node [circle, draw=black] {C}
        }
        child {
          node [circle, draw=black] {$ \lor  $}
          child {
            node [circle, draw=black] {B}
            edge from parent node[above left] {$ \lnot  $}
          }
          child {
            node [circle, draw=black] {C}
          }
          edge from parent node[above right] {$ \lnot  $}
        }
      }
    }
    child {
      node [circle, draw=black] {$ \land  $}
      child {
        node [circle, draw=black] {B}
      }
      child {
        node [circle, draw=black] {$ \land  $}
        child {
          node [circle, draw=black] {A}
        }
        child {
          node [circle, draw=black] {C}
        }
        edge from parent node[above right] {$ \lnot  $}
      }
    };
  \end{tikzpicture}

  \rule{0cm}{0cm}

  \begin{tabular}{|c|c|c|c|}
    \hline
    A & B & C & $ A \land (C \lor \lnot (\lnot B \lor C)) \lor B \land \lnot (A \land C) $ \\\hline
    0 & 0 & 0 & 0 \\\hline
    0 & 0 & 1 & 0 \\\hline
    0 & 1 & 0 & 1 \\\hline
    0 & 1 & 1 & 1 \\\hline
    1 & 0 & 0 & 0 \\\hline
    1 & 0 & 1 & 1 \\\hline
    1 & 1 & 0 & 1 \\\hline
    1 & 1 & 1 & 1 \\\hline
  \end{tabular}

  \subsection{Задача 6}
  Последовательно подставляем значения аргументов из таблицы в данные выражения 
  и получаем, что ей соответствуют только (б) и (г).

  Ответ: б, г


  \subsection{Задача 7}
  Последовательно подставляем значения аргументов из таблицы в данные выражения 
  и получаем, что ей соответствуют только (а) и (в).

  Ответ: а, в


  \section{Страницы 218 - 219}

  \subsection{Задача 1}
  Допустим, что Миша сказал только правду, тогда по его словам Миша и Коля не рабивали 
  окно, значит окно разбил Сергей, но в таком случае Коля тоже сказал только правду,
  что не допустимо по условию задачи.

  Допущение о том, что только правду сказал Коля приедет к тому же самому противоречию,
  значит только правду сказал Сергей. Из слов последнего получаем, что окно разбил 
  Миша, а Сергей этого не делал, значит Коля сказал только ложь, значит по условию задачи
  Миша сказал в одной числи предложения правду, а в другой - ложь, что не вызывает противоречий.

  Ответ: Миша

  \subsection{Задача 2}
  Имеем три пары высказываний, в каждой из которох верно только одно:

  \begin{itemize}
    \item $ A_1 $ -- "Наташа заняла первое место"
    \item $ A_2 $ -- "Маша заняла второе место"
    \item $ B_1 $ -- "Люда заняла второе место"
    \item $ B_2 $ -- "Рита заняла четвертое место"
    \item $ C_1 $ -- "Рита заняла третье место"
    \item $ C_2 $ -- "Наташа заняла второе место"
  \end{itemize}

  Допустим, что верно $ A_1 $, тогда ложны $ A_2 $ и $ C_2 $. Поскольку $ C_2 $ не верно,
  то $ C_1 $ истино, а значит ложно $ B_2 $ и истино $ B_1 $.

  Из верных высказываний следует, что Наташа заняла первое место, Рита -- третье,
  а Люда~-- второе. Оставшееся четвертое место уходит Маше.

  Допустим обратное: $ A_1 $ ложно. Тогда $ A_2 $ истино, а $ B_1 $ не верно, значит
  верно $ B_2 $. Из последнего следует, что не верно $ C_1 $, а значит верно $ C_2 $.
  Однако $ C_2 $ противоречит $ A_1 $, значит изначальное допущение было не верно.

  Ответ: Наташа - 1, Люда - 2, Рита - 3, Маша - 4.

  \subsection{Задача 5}
  Построим таблицу, в которой строкам соответствуют роли, а стольбцам -- люди:

  \rule{0cm}{0.3cm}

  \begin{tabular}{|l|c|c|c|}
    \hline
             & Михаил & Сергей & Виктор \\ \hline
    Командир &        &        &        \\ \hline
    Механик  &        &        &        \\ \hline
    Радист   &        &        &        \\ \hline
  \end{tabular}

  \rule{0cm}{0.3cm}


  Допустим, что верно первое утверждение, тогда таблица нанет заполняться так:

  \rule{0cm}{0.3cm}

  \begin{tabular}{|l|c|c|c|}
    \hline
             & Михаил & Сергей & Виктор \\ \hline
    Командир & 1      &        & 1      \\ \hline
    Механик  &        & 1      &        \\ \hline
    Радист   &        &        & 0      \\ \hline
  \end{tabular}

  \rule{0cm}{0.3cm}

  Однако, командиром в команде может быть только один, значит изначальное допущение было неверно.

  Допустим теперь, что верно второе утверждение. Тогда таблица примет такой вид:

  \rule{0cm}{0.3cm}

  \begin{tabular}{|l|c|c|c|}
    \hline
             & Михаил & Сергей & Виктор \\ \hline
    Командир & 0      & 0      & 1      \\ \hline
    Механик  & 1      & 0      & 0      \\ \hline
    Радист   & 0      & 1      & 0      \\ \hline
  \end{tabular}

  \rule{0cm}{0.3cm}

  Противоречий не возникает, но необходимо рассмотреть остальные варианты допущений.

  Пусть верно третье утверждение. Таблица примет такой вид:

  \rule{0cm}{0.3cm}

  \begin{tabular}{|l|c|c|c|}
    \hline
             & Михаил & Сергей & Виктор \\ \hline
    Командир & 0      &        & 1      \\ \hline
    Механик  &        & 1      &        \\ \hline
    Радист   &        &        & 1      \\ \hline
  \end{tabular}

  \rule{0cm}{0.3cm}
  
  Однако, виктор не может быть одновременно и командиром и мехимиком, значит изначальное допущение было неверно.

  Пусть верно последнее утверждение. Таблица примет такой вид:

  \rule{0cm}{0.3cm}

  \begin{tabular}{|l|c|c|c|}
    \hline
             & Михаил & Сергей & Виктор \\ \hline
    Командир & 0      &        & 0      \\ \hline
    Механик  &        & 1      &        \\ \hline
    Радист   &        &        & 0      \\ \hline
  \end{tabular}

  \rule{0cm}{0.3cm}

  Однако, в таком случае Виктор может быть только механиком,
  хотя эта роль уже занята Сергеем, значит изначальное допущение было неверно.

  \rule{0cm}{0.3cm}

  Ответ: Виктор -- командир, Михаил -- Механик, Сергей -- Радист.



  \pagebreak

  \part{Сообщение}
  \section{<<Постфиксная и инфиксная формы записи выражений>>}
  \subsection{Введение}
  Инфиксная нотация знакома и привычна подовляющему большинству населения 
  Земли, потому что именно она является общепринятой и используется повсеместно.
  Однако это не единствеый спозоб записи выражений, помимо других существуют 
  префиксная и постфиксная формы записи. В данном тексте мы подробнее рассмотрим 
  последнюю из них, так же называемую "Обратная польская нотация".

  \subsection{Описание}
  В привычной нам инфиксной форме записи бнарные операторы записываются между 
  их аргументами, а функции записываются перед списком аргументов, окруженных 
  кобками и разделенных запятыми, так, например, произведение синуса трех и суммы 
  четырех и пяти будет записано как "sin(3) * (4 + 5)". Во избежание неоднозначности 
  в этой нотации присутствуют приоритеты операторов указывающие на порядок вычислений,
  а так же скобки, позволяющие при необходимости изменять этот порядок.

  В обратной польской нотации же все функции и бинарные операторы записываются 
  справа от своих аргументов без каких-либо скобок. Так, описаное выше выражение 
  будет записано как "3 sin 4 5 + *". Значительным преимуществом данной нотации 
  является ее одназначность без необходимости введения приоритета операторов и
  скобок, отсутствие которых делает постфикстую форму записи более компактной 
  и обеспчивает гораздо более простой алгоритм ее разбора и вычисления.

  \subsection{Вычисление} 
  Рассмотрим наиболее простой алгоритм вычисления выражения, записанного 
  постфиксной формой записи, использующий стек в качестве изменяемого состояния
  (Предположим, что текст уже разбит на токены):

  \begin{itemize}
    \item Пока есть токены для чтения:
      \begin{itemize}
        \item Читаем один токен
        \item Если это число, то добавляем его в стрек.
        \item Если это функция или оператор, то применям ее/его к необходимому 
          количеству значений, изымаемых из стека. Результат применения запиываем в стек.
      \end{itemize}
    \item Если выражение было записано корректно, то по окончанию списка тоекнов 
      в стеке должно остаться одно число, являющееся результатом вычисления.
  \end{itemize}

  \subsection{Перевод из инфиксной нотации в постфиксную}
  Рассмотрим наболее часто использующийся алгоритм перевода инфиксной нотации в 
  обратную польскую (Предположим, что тест уже разбит на токены). Для его реализации
  используется стек операций.

  Описание алгоритва:
  \begin{itemize}
    \item Пока есть токен для чтения:
      \begin{itemize}
        \item Читаем один токен
        \item Если это число или постфиксная функция, добавляем его к выходному выражению.
        \item Если это префиксная функция, помещаем его в стек.
        \item Если это открывающая скобка, помещаем его в стек.
        \item Если это закрывающая скобка:
          \begin{itemize}
            \item До тех пор, пока верхним элементом стека не станет открывающая 
              скобка:
              \begin{itemize}
                \item Выталкиваем элементы из стека в выходное выражение. 
              \end{itemize}
            \item Если стек закончился раньше, чем встретилась открывающая 
              скобка, то выражение записано некорректно.
						\item Появление непарной скобки также свидетельствует об ошибке.
            \item Открывающая скобка удаляется из стека. 
          \end{itemize}
				\item Если символ является бинарной операцией о1, тогда:
					\begin{enumerate}
						\item
							\begin{itemize}
								\item Пока на вершине стека префиксная функция,
									или операция на вершине стека приоритетнее o1,
									или операция на вершине стека левоассоциативная с приоритетом как у o1:
									\begin{itemize}
										\item выталкиваем верхний элемент стека в выходную строку;
									\end{itemize}
							\end{itemize}
						\item Помещаем операцию o1 в стек.
					\end{enumerate}
      \end{itemize}
		\item По окончанию списка токенов:
			\begin{itemize}
				\item	Если в стеке есть что-либо кроме скобок, значит в выражении не согласованы скобки.
				\item Выталкиваем все токены из стека в выходное выражение. 
			\end{itemize}
	\end{itemize}

	\subsection{Недостатки и ограничения}
	Постфиксная нотация имеет так же и определенные недостатки, главные из которых:
	\begin{itemize}
		\item Неудобство чтеня для подовляющего большинства людей. Как уже было сказано,
			общепринятой является инфиксная нотация.
		\item Отсутствие возможности использовать операции, являющиеся одновременно и унарными,
			и банрными. Так, например, невозможно использовать одновременно для вычитания и для 
			смены знака числа. Для решения этой проблемы необходимо либо использовать разные 
			символы для инарной и бинарной функции, либо выражать бинарную функцию через унарную
			или наоборот. В случае с минусом можно, например, заменить ``-3'' на ``0 - 3''.
	\end{itemize}
\end{document}
