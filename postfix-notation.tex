\section{<<Постфиксная и инфиксная формы записи выражений>>}
\subsection{Введение}
Инфиксная нотация знакома и привычна подовляющему большинству населения 
Земли, потому что именно она является общепринятой и используется повсеместно.
Однако это не единствеый способ записи выражений, помимо других существуют 
префиксная и постфиксная формы. В данном тексте мы подробнее рассмотрим 
последнюю из них, так же называемую "Обратная польская нотация".

\subsection{Описание}
В привычной нам инфиксной форме записи бинарные операторы записываются между 
их аргументами, а функции -- перед списком аргументов, окруженных 
cкобками и разделенных запятыми, так, например, произведение синуса трех и суммы 
четырех и пяти будет записано как "sin(3) * (4 + 5)". Во избежание неоднозначности 
в этой нотации присутствуют приоритеты операторов указывающие на порядок вычислений,
а так же скобки, позволяющие при необходимости изменять этот порядок.

В обратной польской нотации же все функции и бинарные операторы записываются 
справа от своих аргументов без каких-либо скобок. Так, описаное выше выражение 
будет записано как "3 sin 4 5 + *". Значительным преимуществом данной нотации 
является ее одназначность без необходимости введения приоритета операторов и
использования скобок, отсутствие которых делает постфикстую форму записи более 
компактной и обеспчивает гораздо более простой алгоритм ее разбора и вычисления.

\subsection{Вычисление} 
Рассмотрим наиболее простой алгоритм вычисления выражения, записанного 
постфиксной формой записи, использующий стек в качестве изменяемого состояния
(Предположим, что текст уже разбит на токены):

Описание алгоритма:
\begin{itemize}
  \item Пока есть токены для чтения:
    \begin{itemize}
      \item Читаем один токен
      \item Если это число, то добавляем его в стрек.
      \item Если это функция или оператор, то применеям ее/его к необходимому 
        количеству значений, изымаемых из стека. Результат применения запиываем в стек.
    \end{itemize}
  \item Если выражение было записано корректно, то по окончанию списка тоекнов 
    в стеке должно остаться одно число, являющееся результатом вычисления.
\end{itemize}

\subsection{Перевод из инфиксной нотации в постфиксную}
Рассмотрим наболее часто использующийся алгоритм перевода инфиксной нотации в 
обратную польскую (Предположим, что текст уже разбит на токены). Для его реализации
используется стек операций в качастве изменяемого состояния.

Описание алгоритма:
\begin{itemize}
  \item Пока есть токен для чтения:
    \begin{itemize}
      \item Читаем один токен
      \item Если это число или постфиксная функция, добавляем его к выходному выражению.
      \item Если это префиксная функция или открывающая скобка, помещаем его в стек.
      \item Если это закрывающая скобка:
        \begin{itemize}
          \item До тех пор, пока верхним элементом стека не станет открывающая 
            скобка:
            \begin{itemize}
              \item[-] Выталкиваем элементы из стека в выходное выражение. 
            \end{itemize}
          \item Если стек закончился раньше, чем встретилась открывающая 
            скобка, то выражение записано некорректно.
          \item Появление непарной скобки также свидетельствует об ошибке.
          \item Открывающая скобка удаляется из стека. 
        \end{itemize}
      \item Если символ является бинарной операцией O:
        \begin{itemize}
          \item 
            Пока на вершине стека префиксная функция,
            или операция на вершине стека приоритетнее O,
            или операция на вершине стека левоассоциативная с приоритетом как у O:
            \begin{itemize}
              \item[-] выталкиваем верхний элемент стека в выходную строку;
            \end{itemize}
          \item Помещаем операцию O в стек.
        \end{itemize}
    \end{itemize}
  \item По окончанию списка токенов:
    \begin{itemize}
      \item	Если в стеке есть что-либо кроме скобок, значит в выражении не согласованы скобки.
      \item В ином случае выталкиваем все токены из стека в выходное выражение. 
    \end{itemize}
\end{itemize}

\subsection{Недостатки и ограничения}
Постфиксная нотация имеет так же и определенные недостатки, главные из которых:
\begin{itemize}
  \item Неудобство чтения и записи для подовляющего большинства людей. Как уже было сказано,
    общепринятой является инфиксная нотация.
  \item Отсутствие возможности использовать операции, являющиеся одновременно и унарными,
    и бинарными. Так, например, невозможно использовать минус одновременно для вычитания и для 
    смены знака числа. Для решения этой проблемы необходимо либо использовать разные 
    символы для унарной и бинарной функций, либо выражать бинарную функцию через унарную
    или наоборот. В случае с минусом можно, например, заменить ``-3'' на ``0 - 3''
    и уже это выражение переводить в постфиксную нотацию.
\end{itemize}
