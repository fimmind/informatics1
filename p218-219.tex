\section{Страницы 218 - 219}

\subsection{Задача 1}
Допустим, что Миша сказал только правду, тогда по его словам Миша и Коля не рабивали 
окно, значит окно разбил Сергей, но в таком случае Коля тоже сказал только правду,
что не допустимо по условию задачи.

Допущение о том, что только правду сказал Коля приведет к тому же противоречию,
значит только правду сказал Сергей. Из слов последнего получаем, что окно разбил 
Миша, а Сергей этого не делал, значит Коля сказал только ложь, значит по условию задачи
Миша сказал в одной числи предложения правду, а в другой - ложь, что не вызывает противоречий.

Ответ: Миша

\subsection{Задача 2}
Имеем три пары высказываний, в каждой из которох верно только одно:

\begin{itemize}
  \item $ A_1 $ -- "Наташа заняла первое место"
  \item $ A_2 $ -- "Маша заняла второе место"
  \item $ B_1 $ -- "Люда заняла второе место"
  \item $ B_2 $ -- "Рита заняла четвертое место"
  \item $ C_1 $ -- "Рита заняла третье место"
  \item $ C_2 $ -- "Наташа заняла второе место"
\end{itemize}

Допустим, что верно $ A_1 $, тогда ложны $ A_2 $ и $ C_2 $. Поскольку $ C_2 $ не верно,
то $ C_1 $ истино, а значит ложно $ B_2 $, поскольку оно противоречит $ C_1 $, и истино $ B_1 $.

Из верных высказываний следует, что Наташа заняла первое место, Рита -- третье,
а Люда~-- второе. Оставшееся четвертое место уходит Маше.

Допустим обратное: $ A_1 $ ложно. Тогда $ A_2 $ истино, а $ B_1 $ не верно, 
поскольку оно противоречит $ A_2 $, и, как следствие, верно $ B_2 $. Из последнего 
следует, что не верно $ C_1 $, а значит верно $ C_2 $.  Однако $ C_2 $ 
противоречит $ A_1 $, значит обратное допущение не верно.

Ответ: Наташа - 1, Люда - 2, Рита - 3, Маша - 4.

\subsection{Задача 5}
Построим таблицу, в которой строкам соответствуют роли, а стольбцам -- люди:

\rule{0cm}{0.3cm}

\begin{tabular}{|l|c|c|c|}
  \hline
           & Михаил & Сергей & Виктор \\ \hline
  Командир &        &        &        \\ \hline
  Механик  &        &        &        \\ \hline
  Радист   &        &        &        \\ \hline
\end{tabular}

\rule{0cm}{0.3cm}


Допустим, что верно первое утверждение, тогда таблица начнет заполняться так:

\rule{0cm}{0.3cm}

\begin{tabular}{|l|c|c|c|}
  \hline
           & Михаил & Сергей & Виктор \\ \hline
  Командир & 1      &        & 1      \\ \hline
  Механик  &        & 1      &        \\ \hline
  Радист   &        &        & 0      \\ \hline
\end{tabular}

\rule{0cm}{0.3cm}

Однако, командиром в команде может быть только один, значит изначальное допущение неверно.

Допустим теперь, что верно второе утверждение. Тогда таблица примет такой вид:

\rule{0cm}{0.3cm}

\begin{tabular}{|l|c|c|c|}
  \hline
           & Михаил & Сергей & Виктор \\ \hline
  Командир & 0      & 0      & 1      \\ \hline
  Механик  & 1      & 0      & 0      \\ \hline
  Радист   & 0      & 1      & 0      \\ \hline
\end{tabular}

\rule{0cm}{0.3cm}

Противоречий не возникает, и получившееся соответсвия ролей являются ответом, 
но необходимо рассмотреть остальные варианты допущений.

Пусть верно третье утверждение. Часть таблицы примет такой вид:

\rule{0cm}{0.3cm}

\begin{tabular}{|l|c|c|c|}
  \hline
           & Михаил & Сергей & Виктор \\ \hline
  Командир & 0      &        & 1      \\ \hline
  Механик  &        & 1      &        \\ \hline
  Радист   &        &        & 1      \\ \hline
\end{tabular}

\rule{0cm}{0.3cm}

Однако, Виктор не может быть одновременно и командиром и механиком, значит 
изначальное допущение неверно.

Пусть верно последнее утверждение. Чвсть таблицы примет такой вид:

\rule{0cm}{0.3cm}

\begin{tabular}{|l|c|c|c|}
  \hline
           & Михаил & Сергей & Виктор \\ \hline
  Командир & 0      &        & 0      \\ \hline
  Механик  &        & 1      &        \\ \hline
  Радист   &        &        & 0      \\ \hline
\end{tabular}

\rule{0cm}{0.3cm}

Однако, в таком случае Виктор может быть только механиком,
хотя эта роль уже занята Сергеем, значит изначальное допущение неверно.

\rule{0cm}{0.3cm}

Ответ: Виктор -- командир, Михаил -- Механик, Сергей -- Радист.
